
La lógica modal es una forma de llamar a diversos tipos de sistemas formales que tratan de capturar ciertas nociones. La falta de consenso sobre dichas nociones es la razón de existir de diversos sistemas formales. Por lo anterior, no intentaremos definir que es la lógica modal y nos contentaremos con usar el termino para referirnos al estudio de sistemas donde las formulas lógicas del calculo proposicional admiten dos operadores nuevos, los operadores modales $\necesidad$ y $\posibilidad$ (necesidad y posibilidad).

Así pues, las formulas que tendremos en cuenta son las generadas mediante el uso regular de los conectivos $\land$, $\lor$, y los operadores unarios prefijos $\neg$, $\necesidad$ $\posibilidad$. Por ejemplo si $A$ es una formula proposicional regular, entonces $\necesidad A$ y $\posibilidad A$, también son formulas proposicionales y su lectura sera respectivamente \sarcasmo{necesariamente A} y \sarcasmo{posiblemente A}.



