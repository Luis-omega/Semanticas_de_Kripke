\section{Admisibilidad}

Vamos a preparar el terreno para probar que el sistema axiomático para S4 se corresponde con las reglas para el calculo de secuentes que tenemos.

\begin{teo}
  Para toda formula modal A, en S4 se cumple que  
  \begin{equation*}
    w:A, \Gamma \derives \Delta,w:a 
  \end{equation*}
  es derivable.
\end{teo}

\begin{proof}
  La prueba es por inducción sobre las formas de construir una formula moda. El caso de una formula atómica es directamente un secuente inicial.
  Para los conectivos regulares solo es necesario aplicar las reglas proposicionales, basta aplicar primero la regla derecha correspondiente y después la regla izquierda.
  Para los operadores modales hacemos solo la derivación para la necesidad


  \tree{R\necesidad}{
    \tree{L\necesidad}{
      o:A, w:\necesidad A, wRo, \Gamma \derives \Delta, o:A
    }
    {
    w:\necesidad A , wRo, \Gamma \derives \Delta, o:A
    }
  }
  {
    w:\necesidad A, \Gamma \derives \Delta, w:\necesidad A
  }

\end{proof}


\begin{teo}
  Para toda formulas modales A,B en S4 se cumple que  
  \begin{equation*}
    \derives \necesidad(A \implies B) \implies (\necesidad A \implies \necesidad B) 
  \end{equation*}
  es derivable.
\end{teo}



\begin{proof}

  Por el teorema anterior es valido el secuente inicial de la derivación :

  \tree{R\implies}{
    \tree{R \necesidad}{
      \tree{L \necesidad}{
        \tree{L \necesidad}{
          \tree{L \implies}{
            o:A, o:B, o:A, wRo, w\necesidad A, w:\necesidad(A\implies B) \derives o: B
          }{
            o:A\implies B, o:A, wRo, w\necesidad A, w:\necesidad(A\implies B) \derives o: B
          }
        }{
          o:A, wRo, w\necesidad A, w:\necesidad(A\implies B) \derives o: B
        }
      }{
        wRo, w\necesidad A, w:\necesidad(A\implies B) \derives o: B
      }
    }{
      w\necesidad A, w:\necesidad(A\implies B) \derives w: \necesidad B
    }
  }{
    w:\necesidad(A\implies B) \derives w:(\necesidad A \implies \necesidad B)
  }
\end{proof}

Así pues, nos hace falta probar la admisibilidad de la regla de necesidad para poder afirmar que S4 cumple al menos con los axiomas de \K para logica modal. Para probar la admisibilidad, primero probaremos la regla de weakening y para ello nuestra mas poderosa aliada sera la sustitución de etiquetas definida como:

\begin{align*}
  wRo(r/l) &= wRo \, si \, l \neq w \, y l \neq o \\
  wRo(r/w) &= rRo \, si \, w \neq o \\
  wRo(r/o) &= wRr \, si \, w \neq o \\
  wRw(r/w) &= rRr \\
  w:A(r/o) &= w:A \, si \, o \neq w \\
  w:A(r/w) &= r:A
\end{align*}


\begin{teo}
  Si \Gamma \derives \Delta en S4, entonces \Gamma(o/w) \derives \Delta(o/w) también es derivable y lo es con la misma altura.
\end{teo}
\begin{proof}
  Para el caso $n=0$ y si la sustitución no es vacía, los casos en que es un secuente inicial y no la regla de falso, son inmediatos así como también lo es el caso de la regla de falso.
  
  Para $n>0$ hay varios casos.  Primero si la ultima regla aplicada en la derivación es una proposicional distinta a falso, solo hacemos el caso para $L \land$.
  
  \prfsummary{
  }{
    \tree{L \land}{
      w:A, \Gamma \derives \Delta
    }{
      w:B, \Gamma \derives \Delta
    }{
      w:A \land B, \Gamma \derives \Delta
    } 
  }

  Por lo que usando la hipótesis de inducción en $A$ y $B$ y volviendo a aplicar la regla, se obtiene la conclusión deseada. Lo mismo ocurre con las reglas $L\necesidad$ y $R\posibilidad$. 

  De las reglas modales restantes solo consideramos $R\necesidad$. Si la sustitución es vacía terminamos, si no, tenemos dos casos el primero de los cuales tiene otros dos subcasos. El primer caso es que $w$ aparezca en la regla, asumimos que w no es la variable fresca pues de serlo seria inmediato.

  El primer subcaso es si $o$ no es la variable fresca, en cuyo caso repetimos el proceso de las reglas proposicionales. El segundo subcaso donde $o$ es la variable fresca que requiere la regla, entonces podemos conseguir primero una nueva variable fresca $r$ distinta a $o$ y que no aparece en el resto de secuentes, por lo que es simple remplazar $o$ por $r$ en esta regla sin alterar los secuentes, con lo que podemos proceder como antes. 

  El segundo caso es si $w$ no aparece como etiqueta de la formula principal de la regla, por lo que se puede seguir aplicando el mismo proceso que en los anteriores casos.

  Falta pues, probar que las reglas de reflexividad y transitividad tambien admiten la sustitución. 

  El caso de la reflexividad se hace por casos si $w$ aparece como formula principal o no y es analogo.


  La transitividad requiere de considerar casos como en la regla de $R\necesidad$.
\end{proof}

Debido a que tenemos reglas izquierda, derecha y un caso inicial adicional, la regla de weakening viene como cuatro reglas separadas.

\begin{teo}
  some
  some
  \begin{align*}
    \tree{LW}{\Gamma \derives \Delta}{w:A, \Gamma \derives \Delta} \qquad & \tree{RW}{\Gamma \derives \Delta}{\Gamma \derives \Delta, w:A} \\ 
    \tree{$LW_R$}{\Gamma \derives \Delta}{wRo, \Gamma \derives \Delta} \qquad & \tree{$RW_R$}{\Gamma \derives \Delta}{\Gamma \derives \Delta, wRo} 
  \end{align*}
\end{teo}

\begin{proof}
  Las cuatro pruebas son similares, por lo que solo haremos la primera prueba.
  La prueba es por inducción sobre la altura de la derivación.
  Para cualquiera de lo secuentes iniciales la conclusión se sigue de las reglas iniciales. Para la regla de falso también basta usar la regla de falso. 
  Para una derivación de altura $n>0$, solo hacemos el caso para $L \land$

  \tree{L\land}{\prfsummary{s:C, s:B, \Gamma \implies \Delta}}{s:C\land B, \Gamma \implies \Delta}

  Por hipótesis de inducción tenemos una prueba de altura $n-1$ para

  \prfsummary{w:A, s:C, s:B, \Gamma \implies \Delta}

  por lo que aplicando $L \land$ obtenemos la conclusión.

  El caso de las reglas modales se hace utilizando el teorema de sustitución si es necesario, para remplazar las variables que puedan coincidir y aplicando la hipótesis de inducción como el caso anterior.

  Las reglas de reflexividad y transitividad tampoco ofrecen mayor problema.
\end{proof}
