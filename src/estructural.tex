\section{Admisibilidad}

Vamos a preparar el terreno para probar que el sistema axiomático para S4 se corresponde con las reglas para el calculo de secuentes que tenemos.

\begin{teo}
  Para toda formula modal A, en S4 se cumple que  
  \begin{equation*}
    w:A, \Gamma \derives \Delta,w:a 
  \end{equation*}
  es derivable.
\end{teo}

\begin{proof}
  La prueba es por inducción sobre las formas de construir una formula moda. El caso de una formula atómica es directamente un secuente inicial.
  Para los conectivos regulares solo es necesario aplicar las reglas proposicionales, basta aplicar primero la regla derecha correspondiente y después la regla izquierda.
  Para los operadores modales hacemos solo la derivación para la necesidad


  \tree{R\necesidad}{
    \tree{L\necesidad}{
      o:A, w:\necesidad A, wRo, \Gamma \derives \Delta, o:A
    }
    {
    w:\necesidad A , wRo, \Gamma \derives \Delta, o:A
    }
  }
  {
    w:\necesidad A, \Gamma \derives \Delta, w:\necesidad A
  }

\end{proof}


\begin{teo}
  Para toda formulas modales A,B en S4 se cumple que  
  \begin{equation*}
    \derives \necesidad(A \implies B) \implies (\necesidad A \implies \necesidad B) 
  \end{equation*}
  es derivable.
\end{teo}



\begin{proof}

  Por el teorema anterior es valido el secuente inicial de la derivación :

  \tree{R\implies}{
    \tree{R \necesidad}{
      \tree{L \necesidad}{
        \tree{L \necesidad}{
          \tree{L \implies}{
            o:A, o:B, o:A, wRo, w\necesidad A, w:\necesidad(A\implies B) \derives o: B
          }{
            o:A\implies B, o:A, wRo, w\necesidad A, w:\necesidad(A\implies B) \derives o: B
          }
        }{
          o:A, wRo, w\necesidad A, w:\necesidad(A\implies B) \derives o: B
        }
      }{
        wRo, w\necesidad A, w:\necesidad(A\implies B) \derives o: B
      }
    }{
      w\necesidad A, w:\necesidad(A\implies B) \derives w: \necesidad B
    }
  }{
    w:\necesidad(A\implies B) \derives w:(\necesidad A \implies \necesidad B)
  }
\end{proof}


