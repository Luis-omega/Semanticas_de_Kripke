\section{Semántica de Kripke}

Un frame de Kripke es un conjunto W cuyos elementos llamaremos mundos, junto con una relación binaria R sobre los elementos de W que llamamos relación de accesibilidad. 

Una valuación val para un frame de Kripke toma un mundo w y una formula P y devuelve un valor 0 o 1. Adoptamos la notación :

\begin{equation*}
  w \V P \quad iff \quad val(w,P)=1.
\end{equation*}

Y decimos que : La formula P se cumple en el mundo w. Adicionalmente si val(w, P) = 0, podemos escribir $w \nV P$.

Adicionalmente esperamos que una valuación satisfaga:

\begin{align*}
  &\text{Toda variable proposicional tiene un valor asociado por la valuación en cada mundo}. \\
  &w \V A \land B \quad si \, w \V A \, y \, w \V B. \\
  &w \V A \lor B \quad si \, w \V A \, o \, w \V B. \\
  &w \V A \implies B \quad si \, de \, w \V A \, se \, sigue \, w \V B. \\
  &w \V \falso \, para \, ningun \, w. 
\end{align*}


Con estas condiciones toda formula en un mundo tiene un valor asociado y utilizando las reglas de la valuación para la implicación y la constante falso, se cumple $w \nV A$ implica $w \V \neg A$.

Un frame de Kripke junto con una valuación es llamado un modelo de Kripke.

Diremos que una formula A en un frame de Kripke W es valida en W si, para cada evaluación, $w \V A$  para cada mundo w de W.

Hasta ahora ignoramos a la relación binaria que viene con el frame, para la relación definimos el que una formula sea necesaria como:

\begin{equation*}
  w \V \necesidad A \text{ si y solo si para toda } o, \quad o \V A \quad \text{se sigue de } wRo.
\end{equation*}

Con lo anterior acabamos de establecer un significado para la necesidad.

Si necesitamos otra razón para estudiar la lógica modal a través de los modelos de Kripke, esta seria que varios los diversos axiomas que se pueden añadir al sistema \K, son equivalentes a expresar condiciones sobre la relación de accesibilidad R.  Por ejemplo, el axioma \eqref{eqn:4axiom} es equivalente a que la relación de accesibilidad sea transitiva, probemos solo que la transitividad implica el axioma:

Asuma $ w \V \necesidad A $, sean $r, o$ tales que $wRr$ y $rRo$, por hipótesis $wRo$ y $o \V A$. Luego $ r \V \necesidad A$ para todo $r$ y entonces $w \V \necesidad \necesidad A$ .


De la misma forma se puede probar que se cumple el axioma \eqref{eqn:taxiom} si se da la reflexividad sobre R.

En ambos casos es posible probar el reciproco mediante la selección de un mundo arbitrario y una valuación particular para el mundo y la formula.


Por lo tanto los modelos de Kripke nos ofrecen un lugar donde examinar a la lógica S4.
